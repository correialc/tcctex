
\documentclass[
	12pt,				
	oneside,
	a4paper,
	english,
	brazil,
	]{abntex2}

\usepackage{tcc} 		% Customizacoes

% Pacotes fundamentais 
\usepackage{lmodern}			% Fonte Latin Modern
\usepackage[T1]{fontenc}		% Selecao de codigos de fonte.
\usepackage[utf8]{inputenc}		% Codificacao do documento (acentos)
\usepackage{lastpage}			% Usado pela Ficha catalografica
\usepackage{indentfirst}		% Indenta o primeiro paragrafo de cada secao.
\usepackage{color}				% Controle das cores
\usepackage{graphicx}			% Inclusao de graficos
\usepackage{microtype} 			% Melhorias de justificacao
\usepackage[alf,abnt-emphasize=bf]{abntex2cite}	% Citacoes padrao ABNT
\usepackage{lipsum}				% Geracao automatica de textos de exemplo
% ---
	
% Pacotes de citações
\usepackage[brazilian,hyperpageref]{backref}	 
\usepackage[alf]{abntex2cite}	% Citações padrão ABNT

% Informações de dados para CAPA e FOLHA DE ROSTO
\titulo{\Large{Titulo Épico}}
\autor{Autor(a) Fantástico(a)}
\local{Cidade da instituição}
\data{AAAA}

\instituicao{Universidade Show de Bola}
\departamento{Departamento de Pesquisas Fantásticas}
\filiacao{Nome do meu curso super blaster}
\orientador{Prof. Orientador(a) Inteligentão(ona)}

\preambulo{Trabalho de Conclusão de Curso apresentado ao Curso de blá...blá... como requisito parcial à obtenção do título de blá...blá...}
% ---

% informações do PDF
\makeatletter
\hypersetup{
     	%pagebackref=true,
		pdftitle={\@title}, 
		pdfauthor={\@author},
    	pdfsubject={\imprimirpreambulo},
	    pdfcreator={LaTeX with abnTeX2},
		pdfkeywords={abnt}{latex}{abntex}{abntex2}{relatório técnico}, 
		bookmarksdepth=4
}
\makeatother
% --- 

% O tamanho do parágrafo é dado por:
\setlength{\parindent}{1.3cm}

% Controle do espaçamento entre um parágrafo e outro:
\setlength{\parskip}{0.2cm}  % tente também \onelineskip

% Compila o índice
\makeindex


% Início do documento
\begin{document}


\selectlanguage{brazil}

% Retira espaço extra obsoleto entre as frases.
\frenchspacing 

% ----------------------------------------------------------
% ELEMENTOS PRÉ-TEXTUAIS
% ----------------------------------------------------------
\pretextual

\imprimircapa
\imprimirfolhaderosto*

% inserir lista de ilustrações
\pdfbookmark[0]{\listfigurename}{lof}
\listoffigures*
\cleardoublepage
% ---

% inserir lista de tabelas
\pdfbookmark[0]{\listtablename}{lot}
\listoftables*
\cleardoublepage
% ---

% ---
% inserir o sumario
% ---
\pdfbookmark[0]{\contentsname}{toc}
\tableofcontents*
\cleardoublepage
% ---

% ----------------------------------------------------------
% ELEMENTOS TEXTUAIS
% ----------------------------------------------------------
\textual

% ------------------
% ELEMENTOS TEXTUAIS
% ------------------
\textual

\include{introducao/introducao}
\include{fundamentacao/fundamentacao}
\include{metodologia/metodologia}
\include{resultados/resultados}
\include{conclusao/conclusao}

% ----------------------------------------------------------
% ELEMENTOS PÓS-TEXTUAIS
% ----------------------------------------------------------
\postextual

% Referências bibliográficas
\bibliography{referencias}

\end{document}
